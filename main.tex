\documentclass[12pt]{article}
\usepackage[utf8]{inputenc}

\usepackage[margin=1in]{geometry}
\usepackage{lipsum}

\usepackage[backend=biber,style=ieee]{biblatex}
\addbibresource{sources.bib}

\usepackage{titling}
\newcommand{\subtitle}[1]{%
	\posttitle{%
		\par\end{center}
	\begin{center}\large#1\end{center}
	\vskip0.1em}}%

\usepackage[shortlabels]{enumitem}

\title{Reflection \& Synthesis \\
Clean Air Act}
\subtitle{PEGN 430A}
\author{Tyler Singleton}
\date{31 March 2022}

\begin{document}
\maketitle

\newpage
\setlength{\parindent}{0pt}

% --- Questions Section --- %
\textbf{Part I} \\

% Question 1
\textbf{1. What is a hazardous air pollutant (HAP)? Give three examples.} \\
In 42 U.S.C. \textsection 7412(b)(1), congress has defined a list of hazardous air pollutants. Three examples are: Cyanide Compounds, Glycol ethers, and Fine mineral fibers. The Act further describes additions to the list can be added if an air pollutant is identified as hazardous to human and environmental health in 7412(b)(2). \\

% Question 2
\textbf{2. Describe the national ambient air-quality standards (NAAQS).} \\
The NAAQS is defined in 42 U.S.C. \textsection 7409. From 7409(a-b), the NAAQS is composed to a primary and secondary air quality standard, and gives the Administrator the power to promulgate ambient air-quality standards based on their safety to public health and welfare. 7409(d)(1) requires the Administrator to review such criteria every five years to administer new standards as appropriate. \\

% Question 3
\textbf{3. Define ``major source'' of air pollutants.  Give three examples.} \\
From 42 U.S.C. \textsection 7412(a)(1), a ``major source'' is defined to be either a stationary source (as defined in \textsection 7602(z)) or a group of stationary sources that emits 10 or more tons per year of a single hazardous air pollutant and/or a combination of hazardous air pollutants with an emission of 25 or more tons. Three examples of a major source are: factories, refineries, and power plants \cite{epa}. \\

% Question 4
\textbf{4. What is the concept of prevention of significant deterioration (PSD)?} \\
U.S.C. \textsection 7470 outlines the concept as protecting public health and welfare from exposure to air pollutants, to protect and improve air quality of federal land and historic significance, to promote economic growth that is conscious of clean air resources, to prevent pollution sources from deteriorating air quality across state boundaries, and assist in permits. \\

% Question 5
\textbf{5. What is a Title V Operating Permit and when is it used?} \\
A Title V Operation Permit is regulated in 42 U.S.C. \textsection 7761a. The permit allows for the operation of major sources and minor sources for emission control. It is used to monitor and control emissions to protect air quality. \\

\textbf{Part II} \\

\textbf{1. True or False. A ``major source'' is defined as a stationary or group of stationary sources that emmits more than 10 or more tons of a single hazardous air pollutant and/or 10 or more tons of a combination of hazardous air pollutants?} \\

\textbf{False:}
A ``major source'' is defined as a stationary or group of stationary sources that emmits more than 10 or more tons of a single hazardous air pollutant and/or \textbf{25} or more tons of a combination of hazardous air pollutants. \\

\textbf{2. Select from the following what is NOT a concept of Prevention of Significant Deterioration (PSD) from 42 U.S.C. \textsection 7470.} 

\begin{enumerate}[a.]
    \item ``To protect public health and welfare from any actual or potential adverse effects which in the Administrator's judgement may reasonably be anticipate[d]...''
    \item ``to preserve, protect, and enhance the air quality in national parks, national wilderness areas, national monuments, national seashores, and other areas of special national or regional natural, recreational, scenic, or historic value; ''
    \item ``to provide technical and financial assistance to foreign governments in connection with the development and execution of their air pollution prevention and control programs;''
\end{enumerate}

\textbf{Answer: C} \\
`` to provide technical and financial assistance to \textbf{State and local governments} in connection with the development and execution of their air pollution prevention and control programs; and''

\newpage
\printbibliography

\end{document}
