\documentclass[12pt]{article}
\usepackage[utf8]{inputenc}

\usepackage[margin=1in]{geometry}
\usepackage{lipsum}

\usepackage[backend=biber,style=ieee]{biblatex}
\addbibresource{sources.bib}

\usepackage{titling}
\newcommand{\subtitle}[1]{%
	\posttitle{%
		\par\end{center}
	\begin{center}\large#1\end{center}
	\vskip0.1em}}%

\usepackage[shortlabels]{enumitem}

\title{Reflection \& Synthesis \\
Clean Air Act}
\subtitle{PEGN 430A}
\author{Tyler Singleton}
\date{31 March 2022}

\begin{document}
\maketitle

\newpage
\setlength{\parindent}{0pt}

% --- Questions Section --- %
\textbf{Questions} \\

% Question 1
\textbf{1. What is a hazardous air pollutant (HAP)? Give three examples.} \\
In 42 U.S.C. \textsection 7412(b)(1), congress has defined a list of hazardous air pollutants. Three examples are: Cyanide Compounds, Glycol ethers, and Fine mineral fibers. The Act further describes additions to the list can be added if an air pollutant is identified as hazardous to human and environmental health in 7412(b)(2). \\

% Question 2
\textbf{2. Describe the national ambient air-quality standards (NAAQS).} \\
The NAAQS is defined in 42 U.S.C. \textsection 7409. From 7409(a-b), the NAAQS is composed to a primary and secondary air quality standard, and gives the Administrator the power to promulgate ambient air-quality standards based on their safety to public health and welfare. 7409(d)(1) requires the Administrator to review such criteria every five years to administer new standards as appropriate. \\

% Question 3
\textbf{3. Define ``major source'' of air pollutants.  Give three examples.} \\
From 42 U.S.C. \textsection 7412(a)(1), a ``major source'' is defined to be either a stationary source (as defined in \textsection 7602(z)) or a group of stationary sources that emits 10 or more tons per year of a single hazardous air pollutant and/or a combination of hazardous air pollutants with an emission of 25 or more tons. Three examples of a major source are: factories, refineries, and power plants \cite{epa}. \\

% Question 4




\newpage
\printbibliography

\end{document}
